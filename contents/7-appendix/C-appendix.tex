This appendix gives some examples of equations that are used throughout this thesis.
\section{A simple example}
The following example is adapted from Figure 1 of the documentation for the package nomencl (\url{https://ctan.org/pkg/nomencl}).
\begin{equation}\label{eq:mainEq}
a=\frac{N}{A}
\end{equation}
\nomenclature{$a$}{The number of angels per unit area\nomrefeq}%       %% include the equation number in the list
\nomenclature{$N$}{The number of angels per needle point\nomrefpage}%  %% include the page number in the list
\nomenclature{$A$}{The area of the needle point}%
The equation $\sigma = m a$%
\nomenclature{$\sigma$}{The total mass of angels per unit area\nomrefeqpage}%
\nomenclature{$m$}{The mass of one angel}
follows easily from \Cref{eq:mainEq}.

\section{An even simpler example}
The formula for the diameter of a circle is shown in \Cref{eq:secondEq} area of a circle in \cref{eq:thirdEq}.
\begin{equation}\label{eq:secondEq}
D_{circle}=2\pi r
\end{equation}
\nomenclature{$D_{circle}$}{The diameter of a circle\nomrefeqpage}%
\nomenclature{$r$}{The radius of a circle\nomrefeqpage}%

\begin{equation}\label{eq:thirdEq}
A_{circle}=\pi r^2
\end{equation}
\nomenclature{$A_{circle}$}{The area of a circle\nomrefeqpage}%

Some more text that refers to \eqref{eq:thirdEq}.
\fi  %% end of nomenclature example