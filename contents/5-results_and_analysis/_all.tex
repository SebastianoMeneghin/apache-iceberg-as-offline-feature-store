\engExpl{Sometimes this is split into two chapters.\\Keep in mind: How you are going to evaluate what you have done? What are your metrics?\\Analysis of your data and proposed solution\\Does this meet the goals which you had when you started?}

In this chapter, we present the results and discuss them.

\section{Major results}

Some statistics of the delay measurements are shown in Table~\ref{tab:delayMeasurements}.
The delay has been computed from the time the GET request is received until the response is sent.

\begin{table}[!ht]
  \begin{center}
    \caption{Delay measurement statistics}
    \label{tab:delayMeasurements}
    \begin{tabular}{l|S[table-format=4.2]|S[table-format=3.2]} % <-- Alignments: 1st column left, 2nd middle and 3rd right, with vertical lines in between
      \textbf{Configuration} & \textbf{Average delay (ns)} & \textbf{Median delay (ns)}\\
      \hline
      1 & 467.35 & 450.10\\
      2 & 1687.5 & 901.23\\
    \end{tabular}
  \end{center}
\end{table}

Table \ref{tab:ping_results} shows the measurement of round trip times from four hosts to and from a server.
\begin{table}[ht!]
\caption[RTT for 4 hosts]{Result for the ping measurements of RTT for 4 hosts} 
\label{tab:ping_results}
\vspace{1em}
\centering
\begin{tabular}{l *{4}{S[table-format=2.3]}}
{} & \multicolumn{4}{c}{host to server RTT in ms} \\
\cmidrule{2-5}
Host & \multicolumn{1}{c}{min}  & \multicolumn{1}{c}{avg} & \multicolumn{1}{c}{max} & \multicolumn{1}{c}{mdev} \\
\midrule
h1 & 5.625 & 5.625 & 5.625 & 0.0 \\
h2 & 2.909 & 2.909 & 1.909 & 0.0 \\
h3 & 5.007 & 5.007 & 5.007 & 0.0 \\
h4 & 2.308 & 2.308 & 2.308 & 0.0 \\
\midrule
\end{tabular}
\end{table}
\FloatBarrier

\sweExpl{Fördröj mätstatistik}
\sweExpl{Konfiguration | Genomsnittlig fördröjning (ns) | Median fördröjning (ns)}

Figure \ref{fig:processing_vs_payload_length} shows an example of the performance as measured in the experiments.

\begin{figure}[!ht]
% GNUPLOT: LaTeX picture
\setlength{\unitlength}{0.240900pt}
\ifx\plotpoint\undefined\newsavebox{\plotpoint}\fi
\begin{picture}(1500,900)(0,0)
\sbox{\plotpoint}{\rule[-0.200pt]{0.400pt}{0.400pt}}%
\put(171.0,131.0){\rule[-0.200pt]{4.818pt}{0.400pt}}
\put(151,131){\makebox(0,0)[r]{ 1.5}}
\put(1419.0,131.0){\rule[-0.200pt]{4.818pt}{0.400pt}}
\put(171.0,212.0){\rule[-0.200pt]{4.818pt}{0.400pt}}
\put(151,212){\makebox(0,0)[r]{ 2}}
\put(1419.0,212.0){\rule[-0.200pt]{4.818pt}{0.400pt}}
\put(171.0,292.0){\rule[-0.200pt]{4.818pt}{0.400pt}}
\put(151,292){\makebox(0,0)[r]{ 2.5}}
\put(1419.0,292.0){\rule[-0.200pt]{4.818pt}{0.400pt}}
\put(171.0,373.0){\rule[-0.200pt]{4.818pt}{0.400pt}}
\put(151,373){\makebox(0,0)[r]{ 3}}
\put(1419.0,373.0){\rule[-0.200pt]{4.818pt}{0.400pt}}
\put(171.0,454.0){\rule[-0.200pt]{4.818pt}{0.400pt}}
\put(151,454){\makebox(0,0)[r]{ 3.5}}
\put(1419.0,454.0){\rule[-0.200pt]{4.818pt}{0.400pt}}
\put(171.0,534.0){\rule[-0.200pt]{4.818pt}{0.400pt}}
\put(151,534){\makebox(0,0)[r]{ 4}}
\put(1419.0,534.0){\rule[-0.200pt]{4.818pt}{0.400pt}}
\put(171.0,615.0){\rule[-0.200pt]{4.818pt}{0.400pt}}
\put(151,615){\makebox(0,0)[r]{ 4.5}}
\put(1419.0,615.0){\rule[-0.200pt]{4.818pt}{0.400pt}}
\put(171.0,695.0){\rule[-0.200pt]{4.818pt}{0.400pt}}
\put(151,695){\makebox(0,0)[r]{ 5}}
\put(1419.0,695.0){\rule[-0.200pt]{4.818pt}{0.400pt}}
\put(171.0,776.0){\rule[-0.200pt]{4.818pt}{0.400pt}}
\put(151,776){\makebox(0,0)[r]{ 5.5}}
\put(1419.0,776.0){\rule[-0.200pt]{4.818pt}{0.400pt}}
\put(171.0,131.0){\rule[-0.200pt]{0.400pt}{4.818pt}}
\put(171,90){\makebox(0,0){ 0}}
\put(171.0,756.0){\rule[-0.200pt]{0.400pt}{4.818pt}}
\put(298.0,131.0){\rule[-0.200pt]{0.400pt}{4.818pt}}
\put(298,90){\makebox(0,0){ 10}}
\put(298.0,756.0){\rule[-0.200pt]{0.400pt}{4.818pt}}
\put(425.0,131.0){\rule[-0.200pt]{0.400pt}{4.818pt}}
\put(425,90){\makebox(0,0){ 20}}
\put(425.0,756.0){\rule[-0.200pt]{0.400pt}{4.818pt}}
\put(551.0,131.0){\rule[-0.200pt]{0.400pt}{4.818pt}}
\put(551,90){\makebox(0,0){ 30}}
\put(551.0,756.0){\rule[-0.200pt]{0.400pt}{4.818pt}}
\put(678.0,131.0){\rule[-0.200pt]{0.400pt}{4.818pt}}
\put(678,90){\makebox(0,0){ 40}}
\put(678.0,756.0){\rule[-0.200pt]{0.400pt}{4.818pt}}
\put(805.0,131.0){\rule[-0.200pt]{0.400pt}{4.818pt}}
\put(805,90){\makebox(0,0){ 50}}
\put(805.0,756.0){\rule[-0.200pt]{0.400pt}{4.818pt}}
\put(932.0,131.0){\rule[-0.200pt]{0.400pt}{4.818pt}}
\put(932,90){\makebox(0,0){ 60}}
\put(932.0,756.0){\rule[-0.200pt]{0.400pt}{4.818pt}}
\put(1059.0,131.0){\rule[-0.200pt]{0.400pt}{4.818pt}}
\put(1059,90){\makebox(0,0){ 70}}
\put(1059.0,756.0){\rule[-0.200pt]{0.400pt}{4.818pt}}
\put(1185.0,131.0){\rule[-0.200pt]{0.400pt}{4.818pt}}
\put(1185,90){\makebox(0,0){ 80}}
\put(1185.0,756.0){\rule[-0.200pt]{0.400pt}{4.818pt}}
\put(1312.0,131.0){\rule[-0.200pt]{0.400pt}{4.818pt}}
\put(1312,90){\makebox(0,0){ 90}}
\put(1312.0,756.0){\rule[-0.200pt]{0.400pt}{4.818pt}}
\put(1439.0,131.0){\rule[-0.200pt]{0.400pt}{4.818pt}}
\put(1439,90){\makebox(0,0){ 100}}
\put(1439.0,756.0){\rule[-0.200pt]{0.400pt}{4.818pt}}
\put(171.0,131.0){\rule[-0.200pt]{0.400pt}{155.380pt}}
\put(171.0,131.0){\rule[-0.200pt]{305.461pt}{0.400pt}}
\put(1439.0,131.0){\rule[-0.200pt]{0.400pt}{155.380pt}}
\put(171.0,776.0){\rule[-0.200pt]{305.461pt}{0.400pt}}
\put(30,453){\rotatebox{-270}{\makebox(0,0){Processing time (ms)}}}
\put(805,29){\makebox(0,0){Payload size (bytes)}}
\put(868.0,131.0){\rule[-0.200pt]{0.400pt}{84.074pt}}
\put(995.0,131.0){\rule[-0.200pt]{0.400pt}{98.287pt}}
\put(1173.0,131.0){\rule[-0.200pt]{0.400pt}{118.041pt}}
\put(1325.0,131.0){\rule[-0.200pt]{0.400pt}{134.904pt}}
\put(1350.0,131.0){\rule[-0.200pt]{0.400pt}{137.795pt}}
\put(1439.0,131.0){\rule[-0.200pt]{0.400pt}{155.380pt}}
\end{picture}
\caption[A GNUplot figure]{Processing time vs. payload length}\vspace{0.5cm}
\label{fig:processing_vs_payload_length}
\end{figure}
\FloatBarrier		

Given these measurements, we can calculate our processing bit rate as the inverse of the time it takes to process an additional byte divided by 8 bits per byte:

\[
	\text{bit rate} = \frac{1}{\frac{\text{time}_{\text{byte}}}{8}} = 20.03 \quad kb/s
\] 

\Cref{tab:majorMarkupLMDetailedResult} shows another table in which some values have been set in bold (using \textbackslash B) to emphasize them. Note how the \texttt{S} formatting has been modified so that it considers the weight of the characters and this is able to decimal align even these hold-faced numbers with the numbers in the column above them.

\begin{table}[!ht]
    \centering
    \caption{Median values of sandwich attributes}
    \label{tab:majorMarkupLMDetailedResult}
    \begin{tabular}{l *{2}{S[detect-weight,mode=text,table-format=3.2]}}
        & \multicolumn{2}{c}{\textbf{sites}}\\
        \cmidrule{2-3}
        \textbf{Attribute} & \textbf{A} & \textbf{B} \\
        \midrule
        price (in SEK) & 36.5 & 71.3 \\
        protean (g) & 97.2 & 100.0 \\
        salt (mg) & 9.7 & 9.3 \\
        \hline
        \textbf{Average customer rating in \%} & \B 82.2 & \B 89.9 \\
        \midrule
    \end{tabular}
\end{table}
\FloatBarrier


\Needspace*{4\baselineskip}
\Cref{fig:stackedrust} shows a stacked bar chart using pgfplots. It illustrates how easy it is to take a set of data and make a stacked bar plot. One of the features is the shifted values -- this is very useful when the bar itself is too small to put the value into.

\pgfplotstableread{
Label Numbers  Refs  Struct/Enum  Heap  Arrays
cratesio 70.04 19.83 8.31 1.3 0.52
librs 49.26 30.49 10.80 7.92 1.53
rustc 55.01 24.80 11.54 6.16 2.49
}\testdata


\pgfkeys{
    /pgf/number format/.cd,
    fixed,
    fixed zerofill,
    precision=2
}
\begin{figure}[ht!]
    \centering
    \scalebox{0.9}{
    \begin{tikzpicture}
    \begin{axis}[
        ybar stacked,
        %reverse legend,
        reverse legend=false,
        %https://tex.stackexchange.com/questions/88892/pgfplots-bar-plot-spacing-inbetween-bars
        enlarge x limits=0.4,
	    bar width=45pt,
        /pgfplots/nodes near coords*/.append style={
        every node near coord/.style={
            color=black,
            font=\small,
            name=X,
%            shift={    
%                (50pt,25pt)
%                },
            xshift={50pt},
                yshift ={
                ifthenelse((\plotnum == 4), 30pt,20pt)},
            },
            scatter/@post marker code/.append code={
                \node(Y){};
                \draw(X)--(Y.center);
            }
        },
	    nodes near coords,
        bar shift=5pt,
        ymin=0,
        ymax=115,
        xtick=data,
        width=1\textwidth,
        legend style={draw=none},
        legend image post style={scale=2.0},
        legend style={
            at={(0.5,-0.2)},
            anchor=north,
            legend columns=-2,
            font=\large,
            %mark size=20pt,
        },
        ylabel=Percentage points (\%),
        xticklabels from table={\testdata}{Label},
        xticklabel style={rotate=30},
    ]
    \addplot  table [y=Numbers, meta=Label, x expr=\coordindex] {\testdata};
    \addlegendentry{Numbers}
    \addplot table [y=Refs, meta=Label, x expr=\coordindex] {\testdata};
    \addlegendentry{Refs}
    \addplot  table [y=Struct/Enum, meta=Label, x expr=\coordindex] {\testdata};
    \addlegendentry{Struct/Enum}
    \addplot  table [y=Heap, meta=Label, x expr=\coordindex] {\testdata};
    \addlegendentry{Heap}
    \addplot  table [y=Arrays, meta=Label, x expr=\coordindex] {\testdata};
    \addlegendentry{Arrays}
    \end{axis}
    \end{tikzpicture}}
\caption{Rust types distribution for the compiler, crates.io, and lib.rs.
(percentage) - appears here with the permission of the author - see the thesis at \url{https://urn.kb.se/resolve?urn=urn\%3Anbn\%3Ase\%3Akth\%3Adiva-332124}}
\label{fig:stackedrust}
\end{figure}
\FloatBarrier



\section{Reliability Analysis}

\section{Validity Analysis}

\cleardoublepage
\chapter{Discussion}
\label{ch:discussion}
\generalExpl{This can be a separate chapter or a section in the previous chapter.}