\generalExpl{Often the problem and the problem owner come from industry where a specific solution to a specific problem is desired. This is often “too narrowly” defined and often provides a “too narrow” solution for the result to be interesting from a more general engineering perspective and with “new” experiences as a result. Consider together with the project stakeholders (student, problem owner and academia) how the current problem/proposal could be used to investigate some engineering aspect and whose results could provide new or complementary experience to the engineering community and science. \\The experience comes from a question that the thesis tries to answer with previous and other experience, own or modified methods that give a result that can be used to discuss an answer to the research question.\\This paragraph should thus, in addition to the original “narrow” problem, contain what is to be investigated to create new engineering experience and/or science.}

\engExpl{The first paragraph after a heading is not indented, all of the
  subsequent paragraphs have their first line indented.}
  
This chapter describes the specific problem that this thesis addresses, the context of the problem, the goals of this thesis project, and outlines the structure of the thesis.\\

\generalExpl{Give a general introduction to the area. (Remember to use appropriate references in this and all other sections.)}

% One can use either biblatex or bibtex - set as the option for the document at the top of this file
\ifbiblatex
\engExpl{We use the \emph{biblatex} package to handle our references.  We
use the command \texttt{parencite} to get a reference in parenthesis, like
this \textbackslash parencite\{heisenberg2015\} resulting in \parencite{heisenberg2015}.  It is also possible to include the author as part of the sentence using \texttt{textcite}, like talking about the work of \textbackslash textcite\{einstein2016\} resulting in \textcite{einstein2016}.\\
This also means that you have to change the include files to include biblatex and change the way that the \texttt{reference.bib} file is included.}
\else
\engExpl{We use the \emph{bibtex} package to handle our references.  We, therefore,
use the command \textbackslash cite\{farshin\_make\_2019\}. For example, Farshin, \etal described how to improve LLC
cache performance in \cite{farshin_make_2019} in the context of links running
at \qty{200}{Gbps}.}
\fi

\engExpl{Use the glossaries package to help yourself and your readers.
Add the acronyms and abbreviations to lib/acronyms.tex. Some examples are shown below:}
In this thesis, we will examine the use of \glspl{LAN}. In this thesis, we will
assume that \glspl{LAN} include \glspl{WLAN}, such as \gls{WiFi}.