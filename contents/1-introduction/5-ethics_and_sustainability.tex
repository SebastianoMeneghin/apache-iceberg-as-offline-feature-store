This research project focusses on software, namely the development of more efficient data-intensive processing pipelines applicable in \gls{ML} and \gls{NN} training. The Green Software Foundation~\footnote{Foundation's website available at \url{https://greensoftware.foundation/}} states that software may be "part of the climate problem or part of the climate solution." Green software is described as software that minimises its environmental effect by using fewer physical resources and less energy, while optimising energy use to employ lower-carbon sources \cite{WhatGreenSoftware2021}. In the realm of \gls{ML} and \gls{NN} training, minimising training duration--and therefore the read and write latencies associated to the dataset--has been proved to significantly decrease carbon emissions~\cite{pattersonCarbonEmissionsLarge2021,pattersonCarbonFootprintMachine2022}. This thesis minime latency and thus enhance data throughput for reading and writing on Iceberg tables on \gls{HopsFS}. This objective adheres to the essential principles of green software by reducing \gls{CPU} utilisation time relative to the prior system. Minimising \gls{CPU} utilisation time lowers energy consumption, resulting in a reduced carbon impact. This as a concequence, enhance energy efficiency of data-intensive computer pipelines, which are extensively used in \gls{ML} and \gls{NN} training.

For the same reason, this project supports two \glspl{SDG}~\footnote{\glspl{SDG} website available at \url{https://sdgs.un.org/}}, namely Affordable and Clean Energy (Goal 7), and Industry Innovation and Infrastructure (Goal 9). In particular, it tackles objectives 7.3, "Double the improvement in energy efficiency", and 9.4, "Upgrade all industries and infrastructures for sustainability".

\begin{figure}
    \centering
    \begin{subfigure}[b]{0.5\linewidth}
        \centering
        \includegraphics[width=0.6\linewidth]{figures/1-introduction/SDG_icons-07.png} 
        \label{fig:sdg07}
    \end{subfigure}\hfill
    \begin{subfigure}[b]{0.5\linewidth}
        \centering
        \includegraphics[width=0.6\linewidth]{figures/1-introduction/SDG_icons-09.png}
        \label{fig:sdg09}
	\end{subfigure}
	\caption[Sustainable Development Goals supported by this thesis]{Illustrations of the \glstext{SDG} supported by this thesis.}
	\label{fig:sdgs}
\end{figure}

Furthermore, the experiments created to satisfy \gls{G}2--3 has been carefully designed to minimize the number of trails necessary for statistical relevance, pursuing resource effiency. All the data used for this thesis are thoroughly explained, and all the scripts made available, to best address reproducibility of this project.