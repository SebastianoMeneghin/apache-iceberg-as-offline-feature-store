The limitations of this study primarily stem from constraints in resources, both in terms of time and computational resources, and from in scope, which is closely tied to the defined industrial use case, described in Section~\ref{subsec:method_use_case}.

This research focuses on reducing the read and write latency of the Hopsworks offline feature store, which dictated the choice of technology for implementation and experimentation. As a result, the findings are intrinsically influenced by the specific tools and infrastructure used, particularly given the collaboration with the company developing this technology, lowering their generality. Furthermore, the thesis' scope was shaped by the predefined use case, described in Section~\ref{subsec:method_use_case}, which determined the \gls{CPU} and data loads used in the experiments. While this helped establish a clear boundary for the research contributions, it also limits the broader applicability of the results, needing further research to validate these findings in different contexts.

Despite access to substantial computational resources, their availability was restricted due the shared nature of the development environment, where other critical workloads had to be prioritized. Additionally, time constraints limited the number of experiments to reach a 95\% confidence interval. All experiments were run fifty times, which significantly increased the time required to perform all experiments, reaching 350 hours.