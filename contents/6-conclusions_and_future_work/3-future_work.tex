The findings and limitations of this thesis open multiple avenues for future research. The scope of this work was primarily constrained by the specific industrial use case and the available computational resources. Relaxing one or more of these constraints could provide further insights and help generalize the findings across broader applications.

Future experiments should investigate the system's behavior with significantly larger datasets (1B rows, 10B rows, or more). The current study focused on tables up to 60M rows, but exploring a broader data scale could help determine whether there is a threshold where a Spark-based system begins to outperform the proposed alternatives due to its distributed nature. Similarly, additional experiments on different data formats, workloads, and storage solutions--such as \gls{AWS} S3 or \gls{GCS}-- could further validate the results. This thesis was conducted in collaboration with Hopsworks AB, focusing on optimizing the Hopsworks offline feature store. While the findings are valuable within the specific use case context, described in Section \ref{subsec:method_use_case}, testing the proposed approaches on other feature store implementations--such as Databricks or Feast--would help assess their broader applicability. As well, new libraries supporting the \gls{OTF} evaluated in this thesis, like iceberg-rust, should be investigated, as they might further improve performances and scalability, better exploiting local resources and various computational settings.

Lastly, while this thesis focused on PyIceberg and delta-rs as alternatives to the legacy Spark-based system, other \glspl{OTF} are currently under development, such as Apache Paimon, and performing similar experiments and comparisons with these technologies would broaden the knowledge on the possible alternatives to Spark.