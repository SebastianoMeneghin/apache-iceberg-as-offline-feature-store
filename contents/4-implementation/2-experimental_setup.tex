As defined in Section \ref{subsec:experimental_design}, experiments run different system configurations with five tables fifty times per experiment.
The run time of the experiments, i.e., the read and write latency, was measured using two different approaches. This decision is motivated by the need to measure accurate results, while it was impossible to use the most accurate method for measuring across all systems. The initial technique utilizes Python's timeit function, which ensures the process executing the code runs in isolation, providing a reliable approximation of operation latency. As shown in Listing \ref{lst:ch4_timeit}, timeit is implemented by specifying a SETUP\_CODE that executes before the experiment and a TEST\_CODE that is measured during execution, with the resulting time (in seconds) returned by the timeit function. This method was chosen because timeit offers a straightforward interface for executing and timing a concise code snippet, and was used to conduct the experiments about the read operations of legacy system, described in Section~\ref{subsec:back_sys_hudi_read}.

\begin{minipage}{\textwidth}
    \begin{python}[caption={[Measuring latency using Timeit] Timeit usage to measure the time to read from an Iceberg table stored on \gls{HopsFS}.}, label={lst:ch4_timeit}, basicstyle=\small]
    import timeit
    SETUP_CODE='''
    from pyiceberg.catalog import load_catalog'''
        
    TEST_CODE='''
    catalog = load_catalog()
    table   = catalog.load_table("ns.table")
    df      = table.scan().to_arrow()'''
    
    # Measure the execution runtime
    write_result = timeit.timeit(setup  = SETUP_CODE,
                                 stmt   = TEST_CODE,
                                 number = 1          )
    \end{python}
\end{minipage}
\medskip

Running in an isolated Python environment, the timeit approach had two limitations: (1) it was not possible to breakdown the two main steps performed by the legacy system when performing a write operation, (2) it was not possible to separate the timings of the creation of a new catalog and of the read operations in IcedHops, since timeit cannot access variables declared elsewhere (this catalog creation/deletion was necessary at every step, to the machine to perform caching). Thus, for those two cases, see Sections \ref{subsec:back_sys_hudi_write} and \ref{subsec:back_sys_iceberg_read}, a second approach was implemented. This approach consisted in recording the time before and after the script run, using the function time, from the Python standard library called time, and an example of this approach is showed in Listing \ref{lst:ch4_timetime}. The usage of two different approaches was not considered problematic during the experiments, as some trials revealed that the latency measured by the two methods was equal within a 95\% confidence interval.

\begin{minipage}{\textwidth}
    \begin{python}[caption={[Measuring latency using the time difference] A simple time difference approach the time to read from an Iceberg table stored on \gls{HopsFS}.}, label={lst:ch4_timetime}, basicstyle=\small]
    import time
    from pyiceberg.catalog import load_catalog

    catalog = load_catalog()

    before = time.time()
    table  = catalog.load_table("ns.table")
    df     = table.scan().to_arrow()
    after  = time.time()

    reading_time = after - before
    \end{python}
\end{minipage}