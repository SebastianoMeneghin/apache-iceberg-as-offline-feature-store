% The first abstract should be in the language of the thesis.
  \markboth{\abstractname}{}
\begin{scontents}[store-env=lang]
eng
\end{scontents}
%%% The contents of the abstract (between the begin and end of scontents) will be saved in LaTeX format
%%% and output on the page(s) at the end of the thesis with information for DiVA facilitating the correct
%%% entry of the meta data for your thesis.
%%% These page(s) will be removed before the thesis is inserted into DiVA.

\generalExpl{Keep in mind that most of your potential readers are only going to read your \texttt{title} and \texttt{abstract}. This is why the abstract must give them enough information so that they can decide if this document is relevant to them or not. Otherwise, the likely default choice is to ignore the rest of your document.\\
An abstract should stand on its own, i.e., no citations, cross-references to the body of the document, acronyms must be spelled out, \ldots .\\Write this early and revise as necessary. This will help keep you focused on what you are trying to do.}

\begin{scontents}[store-env=abstracts,print-env=true]
Write an abstract that is about 250 and 350 words (1/2 A4-page)  with the following components:
% key parts of the abstract
\begin{itemize}
  \item What is the topic area? 
  \item (optional) Introduces the subject area for the project.
  \item Short problem statement
  \item Why was this problem worth a Master’s thesis project? (\ie, why is the problem both significant and of a suitable degree of difficulty for Master’s thesis project? Why has no one else solved it yet?)
  \item How did you solve the problem? What was your method/insight?
  \item Results/Conclusions/Consequences/Impact: What are your key results/\linebreak[4]conclusions? What will others do based on your results? What can be done now that you have finished - that could not be done before your thesis project was completed?
\end{itemize}
\end{scontents}


\subsection*{Keywords}
\begin{scontents}[store-env=keywords,print-env=true]
% If you set the EnglishKeywords earlier, you can retrieve them with:
\InsertKeywords{english}
% If you did not set the EnglishKeywords earlier then simply enter the keywords here:
% comma separate keywords, such as: Canvas Learning Management System, Docker containers, Performance tuning
\end{scontents}


\textbf{Formatting the keywords}:
\begin{itemize}
  \item The first letter of a keyword should be set with a capital letter and proper names should be capitalized as usual.
  \item Spell out acronyms and abbreviations.
  \item Avoid "stop words" - as they generally carry little or no information.
  \item List your keywords separated by commas (",").
\end{itemize}