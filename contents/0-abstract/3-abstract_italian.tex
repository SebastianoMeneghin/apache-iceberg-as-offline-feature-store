\markboth{\abstractname}{}
\begin{scontents}[store-env=lang]
ita
\end{scontents}



\begin{scontents}[store-env=abstracts,print-env=true]
La crescente necessità di piattaforme per una efficience gestione dei dati per applicazioni di \textit{Machine Learning (ML)} ha portato a un'ampia diffusione dei \textit{feature store}, piattaforme dati centralizzate che supportano \textit{feature engineering}, addestramento di modelli e inferenza. Il \textit{feature store} di Hopsworks ha dimostrato prestazioni superiori rispetto alle sue alternative, sfruttando Apache Hudi e Spark per il \textit{suo offline feature store}. Tuttavia, questo sisteme soffre di un'elevata latenza di scrittura e lettura, anche per piccole quantità di dati (1GB o inferiori). Questa tesi esplora l'uso di Apache Iceberg come \textit{table format} alternativo, integrandolo con HopsFS (distribuzione Hopsworks di HDFS) e la libreria Python PyIceberg per ridurre tali latenze.

Questa ricerca inizia con un'analisi delle possibili strategie di integrazione, documentando i vantaggi e i limiti di ciascun approccio. Successivamente, viene implementata e valutata un'architettura basata su PyIceberg, confrontata con la soluzione esistente basata su Spark e con un'alternative basata su Delta Lake (delta-rs). Esperimenti approfonditi sono stati condotti su tabelle di diverse dimensioni e diverse configurazioni CPU per misurare le prestazioni di scrittura e lettura. I risultati mostrano che PyIceberg riduce significativamente la latenza di scrittura -- da 40 a 140 volte inferiore rispetto al sistema legacy -- e la latenza di lettura -- dal 55\% a 60 volte inferiore. Rispetto a delta-rs, PyIceberg offre una latenza di scrittura inferiore per le tabelle più grandi -- fino a 7 volte minore -- e prestazioni di lettura equivalenti, ma mostra minori vantaggi di scalabilità con l'aumento dei CPU cores (20\% ridotti rispetto a delta-rs).

Questi risultati confermano che alternative ad architetture basate su Spark, per gestire dati su piccola scala, esistono e sono più efficienti, e dunque il sistema sviluppato verrà integrato nel \textit{feature store} di Hopsworks. Inoltre, questa tesi fornisce una base per futuri studi su nuovi \textit{open table format} e strategie da adottare per ottimizzare l'uso delle risorse nelle piattaforme di gestione dei dati.
\end{scontents}



\subsection*{Parole Chiave}
\begin{scontents}[store-env=keywords,print-env=true]
Machine Learning, Feature Store, Apache Iceberg, PyIceberg, Latenza in Lettura/Scrittura, Open Table Formats.
\end{scontents}