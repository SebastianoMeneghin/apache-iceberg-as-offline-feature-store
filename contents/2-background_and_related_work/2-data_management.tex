This section introduces the concept of data mananagement layer, starting from its origin and evolution in to today's technologies. This dissertation includes advantages and limitations of each evolutive step, and describe the functions of \gls{DBMS}. what a data storage is and which typologies of storages exist. The subsections \ref{subsec:hudi_iceberg_delta} explains in details the main \gls{DBMS} technologies of this project, namely Hudi, employed in the legacy version of Hopsworks feature store, Iceberg and Delta Lake, the two alternatives evaluated.

\subsection{Brief history of \glsfmtlongpl{DBMS}}
\label{subsec:history_DBMS}
Since the 2010s, with the advent of Big Data, the data volume, variety, and production velocity have increased exponentially \cite{ederUnstructuredData802008, penceWhatBigData2014}. While on one side, this proved to be of enormous value, on the other this posed several challenges \cite{demchenkoAddressingBigData2012} on data architectures, which had to evolve to cope with these new needs. Data lakehouse technologies, like Hudi, Iceberg and Delta Lake \cite{rajaperumalUberEngineeringIncremental2017,IcebergExamples2024,armbrustDeltaLakeHighperformance2020} are the last step of this evolution. However, to truly understand this tools, it is needed to start from the beginning of \gls{DBMS} evolution.


\subsection{Apache Hudi vs. Apache Iceberg vs. Delta Lake}
\label{subsec:hudi_iceberg_delta}

\subsubsection*{Apache Hudi}
\subsubsection*{Apache Iceberg}
\subsubsection*{Delta Lake}