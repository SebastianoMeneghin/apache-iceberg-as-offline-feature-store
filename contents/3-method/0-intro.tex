This chapter defines three methodologies that will be applied sequentially in this project, answering the two \glspl{RQ} defined in Section~\ref{subsec:research_questions}. Section~\ref{sec:system_integration} defines the system integration process that outputs part of \gls{D}3, i.e., the integration detail. This output will enable the Hudi vs. Iceberg system evaluation defined in Section~\ref{sec:system_evaluation_hudi_iceberg}, which will output \gls{D}1, i.e., the results of the experiments. The Iceberg experiment results (\gls{D}1-partial) will enable the Iceberg vs. Delta Lake system evaluation, defined in Section~\ref{sec:system_evaluation_iceberg_delta}, which will output \gls{D}2, i.e.,the results of the comparative experiments. The analysis of \glspl{D}1--2 will be delivered in \gls{D}3, i.e. this comprehensive thesis report.

It has to be noticed that the system evaluation processes, described in Sections \ref{sec:system_evaluation_hudi_iceberg}-\ref{sec:system_evaluation_iceberg_delta}, are inspired by the system evaluation process of a related work \cite{manfrediReducingReadWrite2024}. That related work was hosted by the same company which hosted this thesis work, Hopsworks AB, and it focused on answering \glspl{RQ} on some intent similar to the ones described in Section \ref{subsec:research_questions}. The related work enabled write and read operations on Delta Lake tables stored on \gls{HopsFS}, and investigated performance differences between that system and the current Hopsworks legacy system. Following similar system evaluation processes will not only allow to answer this project's \glspl{RQ}, but will also enable the reader to fairly compare the three technologies investigated in these two thesis works. For the same reason, it easen the comparative process described in Section \ref{sec:system_evaluation_iceberg_delta}, and will make its results more fair and consistent.