\generalExpl{This chapter is about Engineering-related
  content, Methodologies and Methods.  Use a self-explaining title.\\The
  contents and structure of this chapter will change with your choice of
  methodology and methods.}


\generalExpl{Describe the engineering-related contents (preferably with models) and the research methodology and methods that are used in the degree project.\\
Give a theoretical description of the scientific or engineering methodology  you are going to use and why have you chosen this method. What other methods did you consider and why did you reject them?\\
In this chapter, you describe what engineering-related and scientific skills you are going to apply, such as modeling, analyzing, developing, and evaluating engineering-related and scientific content. The choice of these methods should be appropriate for the problem. Additionally, you should be conscious of aspects relating to society and ethics (if applicable). The choices should also reflect your goals and what you (or someone else) should be able to do as a result of your solution - which could not be done well before you started.
What scientific or engineering methodology did you use and why did you choose this methodology. What other methods did you consider and why you reject them.
What are your objectives (what should you be able to do as a result of your solution - which cannot be done well before you started)
What you are going to do? How to do it? Why are you doing it? For example, if you have implemented an artifact what did you do and why? How will you evaluate it. \\
The purpose of this chapter is to give an overview of the research methodology
used in this thesis. Section~\ref{sec:researchProcess} describes the research process. Section~\ref{sec:researchParadigm} describes the research paradigm in detail. Section~\ref{sec:dataCollection} focuses on the data collection techniques used for this research. Section~\ref{sec:experimentalDesign} describes the experimental
design. Section~\ref{sec:assessingReliability} explains the techniques used to assess the
the reliability and validity of the data collected. Section~\ref{sec:plannedDataAnalysis}
describes the methodology used for the data analysis. Finally, Section~\ref{sec:evaluationFramework}
describes the framework chosen to evaluate xxx.\\
Often a number of follow-up questions can be linked to the research question and problem solving e.g.
(1) What process should be used for the construction of the solution and what process should be linked to it to answer the research question?
(2) How and what results (quantities) should be presented both to report answers to the research question (results chapter in this report) and to report results of the problem solution (prototype, often documents as appendices but which documents and why?)
(3) Which theory/technology should be chosen and used both for the investigation (taxonomy, mathematics, graphs, quantities etc.) and problem solving (UML, UseCases, Java etc.) and why?\\
(4) What do you as a student need to deliver to achieve high quality (minimum requirements) or very high quality of the thesis?\\
(5) The questions link to the following subchapters.
(6) The reasoning is based on the fact that students in the hing program often have to design something for the problem owner and that this must be linked to an interesting engineering question. There is always a dualism between these aspects in the thesis.
}

\section{Research Process}
\label{sec:researchProcess}

Figure~\ref{fig:researchprocess} shows the steps carried out to implement\\
Describe, preferably with an activity diagram (UML?), your research process and development process.  You need to link the academic interest (research process) with the original problem (development process)
this research.\\
Activity diagram from e.g. UML standard


 
\begin{figure}[!ht]
  \begin{center}
    \includegraphics[width=0.5\textwidth]{figures/researchprocess.png}
  \end{center}
  \caption{Research Process}
  \label{fig:researchprocess}
\end{figure}

\generalExpl{Example of using customized item labels.}
Some steps in the process:
\begin{enumerate}[leftmargin=*, label=\textbf{Step \arabic*}, ref=Step \arabic*] %labelindent=1em for indent
    \itemsep0em
    \item \label{x:s1} plan experiment,
    \item \label{x:s2} conduct experiment,
    \item \label{x:s3} analyze data from the experiment, and
    \item \label{x:s4} discuss the results of the analysis.
\end{enumerate}


\section{Research Paradigm}
\label{sec:researchParadigm}
For example\\
Positivistic (what/how does it work?) qualitative case study with a deductive (pre-determined) chosen approach and an inductive (gradually emerging data areas and data) collection of data and experiences.


\section{Data Collection}
\label{sec:dataCollection}
\generalExpl{This should also show that you are aware of the social and ethical concerns that might be relevant to your data collection method.}



\subsection{Sampling}

\subsection{Sample Size}

\subsection{Target Population}
\section[Experimental design/Planned Measurements]{Experimental design and\\Planned Measurements}
\label{sec:experimentalDesign}

\subsection{Test environment/test bed/model}
\generalExpl{Describe everything that someone else would need to reproduce your test environment/test bed/model/… .}

\subsection{Hardware/Software to be used}


\section{Assessing reliability and validity of the data collected}
\label{sec:assessingReliability}


\subsection{Validity of method}
\label{sec:validtyOfMethod}

\generalExpl{How will you know if your results are valid?
Remember that validity is about the \textit{accuracy} of a measurement while reliability is about the \textit{consistency} of the measurement values under the same conditions (\ie repeatability).
}


\subsection{Reliability of method}
\label{sec:reliabilityOfMethod}
\generalExpl{How will you know if your results are reliable? How good are your methods, are there better methods? How can you improve them?}

\subsection{Data validity}
\label{sec:dataValidity}
\generalExpl{How do you know if your results are valid? Are your results accurate?}

\subsection{Reliability of data}
\label{sec:reliabilityOfData}
\generalExpl{How do you know if your results are reliable? How good are your results?}


\section{Planned Data Analysis}
\label{sec:plannedDataAnalysis}

\subsection{Data Analysis Technique}
\label{sec:dataAnalysisTechnique}

\subsection{Software Tools}
\label{sec:softwareTools}


\section{Evaluation framework}
\label{sec:evaluationFramework}
\generalExpl{Method for evaluation, comparison etc. Links to chapters~\ref{ch:resultsAndAnalysis}}


\section{System documentation}
\label{sec:systemDocumentation}
\generalExpl{
With which documents and how should a constructed prototype be documented? These often become appendices to the report and what the problem owner of the original problem (industry) often wants.\\
These annexes often include, and according to some specified standard, requirements documents, architecture documents, design documents, implementation documents, operational documents, test protocols, etc.
If this is going to be a complete document consider putting it in as an appendix, then just put the highlights here.}